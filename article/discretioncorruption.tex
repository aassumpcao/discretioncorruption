\documentclass[11pt]{article}

\usepackage{amssymb,amsmath,amsfonts,eurosym,geometry,ulem,graphicx,caption,color,setspace,sectsty,comment,footmisc,caption,natbib,pdflscape,subfigure,array,hyperref}

\normalem

\singlespacing
\newtheorem{theorem}{Theorem}
\newtheorem{corollary}[theorem]{Corollary}
\newtheorem{proposition}{Proposition}
\newenvironment{proof}[1][Proof]{\noindent\textbf{#1.} }{\ \rule{0.5em}{0.5em}}

\newtheorem{hyp}{Hypothesis}
\newtheorem{subhyp}{Hypothesis}[hyp]
\renewcommand{\thesubhyp}{\thehyp\alph{subhyp}}

\newcommand{\red}[1]{{\color{red} #1}}
\newcommand{\blue}[1]{{\color{blue} #1}}

\newcolumntype{L}[1]{>{\raggedright\let\newline\\arraybackslash\hspace{0pt}}m{#1}}
\newcolumntype{C}[1]{>{\centering\let\newline\\arraybackslash\hspace{0pt}}m{#1}}
\newcolumntype{R}[1]{>{\raggedleft\let\newline\\arraybackslash\hspace{0pt}}m{#1}}

\geometry{left=1.0in,right=1.0in,top=1.0in,bottom=1.0in}

\begin{document}

\begin{titlepage}
\title{Placeholder\thanks{abc}}
\author{Hyunwoo Park\thanks{abc} \and John Smith\thanks{abc}}
\date{\today}
\maketitle
\begin{abstract}
\noindent Placeholder\\
\vspace{0in}\\
\noindent\textbf{Keywords:} key1, key2, key3\\
\vspace{0in}\\
\noindent\textbf{JEL Codes:} key1, key2, key3\\

\bigskip
\end{abstract}
\setcounter{page}{0}
\thispagestyle{empty}
\end{titlepage}
\pagebreak \newpage




\singlespacing


\section{Introduction} \label{sec:introduction}

\section{Literature Review} \label{sec:literature}

\section{Data} \label{sec:data}

\section{Results} \label{sec:result}

\section{Discussions} \label{sec:discussion}

\section{Conclusion} \label{sec:conclusion}



\singlespacing
\setlength\bibsep{0pt}


\citep{AssumpcaotextfindDataDrivenText2018}

\clearpage

\singlespacing

\section*{Tables} \label{sec:tab}



\clearpage

\section*{Figures} \label{sec:fig}

%\begin{figure}[hp]
%  \centering
%  \includegraphics[width=.6\textwidth]{../fig/placeholder.pdf}
%  \caption{Placeholder}
%  \label{fig:placeholder}
%\end{figure}


\clearpage

\section*{Appendix A. Placeholder} \label{sec:appendixa}

Service orders issued by CGU investigated different uses of public
resources in addition to procurement, e.g.~for officials compensation,
for school activities, or for community monitoring of public policies.
The discretion measure proposed here, however, is exclusive to
procurement expenditures made under Law 8,666/93. The ideal dataset for
this study would contain explicit procurement information collected by
CGU auditors, but unfortunately this is not the case. The reporting of
procurement processes is implicit, via descriptions of investigations or
findings of violations to Law 8,666/93. Thus, we isolate service orders
which investigated procurement processes from the rest by implementing
an classification system based on the information retrieval and
natural-language processing literatures.

The system uses each service order's description to identify if it is
procurement-related. In these descriptions, CGU auditors report the
purpose of their investigation, e.g.~whether they are looking into
painkiller purchases, whether the municipality has used the funds within
designated goals, or whether primary school teachers were hired for the
implementation of a school program. Using these textual descriptions as
bag-of-words models, we implement a method similar to that of Hopkins \&
King (2009): we stem and combine unigrams to form search patterns that
identify a service order as procurement-related. There are two broad
types of procurement in Law 8,666/93: (i) ordinary procurement of goods
and services, which we call \emph{purchases}; and (ii) procurement of
goods and services used for public works, which we call \emph{works}.
There are different search patterns for each type.

\bibliographystyle{apa}
\bibliography{library}

\end{document}