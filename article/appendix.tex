\documentclass[]{article}
\usepackage{lmodern}
\usepackage{amssymb,amsmath}
\usepackage{ifxetex,ifluatex}
\usepackage{fixltx2e} % provides \textsubscript
\ifnum 0\ifxetex 1\fi\ifluatex 1\fi=0 % if pdftex
  \usepackage[T1]{fontenc}
  \usepackage[utf8]{inputenc}
\else % if luatex or xelatex
  \ifxetex
    \usepackage{mathspec}
  \else
    \usepackage{fontspec}
  \fi
  \defaultfontfeatures{Ligatures=TeX,Scale=MatchLowercase}
\fi
% use upquote if available, for straight quotes in verbatim environments
\IfFileExists{upquote.sty}{\usepackage{upquote}}{}
% use microtype if available
\IfFileExists{microtype.sty}{%
\usepackage{microtype}
\UseMicrotypeSet[protrusion]{basicmath} % disable protrusion for tt fonts
}{}
\usepackage[margin=1in]{geometry}
\usepackage{hyperref}
\hypersetup{unicode=true,
            pdfborder={0 0 0},
            breaklinks=true}
\urlstyle{same}  % don't use monospace font for urls
\usepackage{longtable,booktabs}
\usepackage{graphicx,grffile}
\makeatletter
\def\maxwidth{\ifdim\Gin@nat@width>\linewidth\linewidth\else\Gin@nat@width\fi}
\def\maxheight{\ifdim\Gin@nat@height>\textheight\textheight\else\Gin@nat@height\fi}
\makeatother
% Scale images if necessary, so that they will not overflow the page
% margins by default, and it is still possible to overwrite the defaults
% using explicit options in \includegraphics[width, height, ...]{}
\setkeys{Gin}{width=\maxwidth,height=\maxheight,keepaspectratio}
\IfFileExists{parskip.sty}{%
\usepackage{parskip}
}{% else
\setlength{\parindent}{0pt}
\setlength{\parskip}{6pt plus 2pt minus 1pt}
}
\setlength{\emergencystretch}{3em}  % prevent overfull lines
\providecommand{\tightlist}{%
  \setlength{\itemsep}{0pt}\setlength{\parskip}{0pt}}
\setcounter{secnumdepth}{5}
% Redefines (sub)paragraphs to behave more like sections
\ifx\paragraph\undefined\else
\let\oldparagraph\paragraph
\renewcommand{\paragraph}[1]{\oldparagraph{#1}\mbox{}}
\fi
\ifx\subparagraph\undefined\else
\let\oldsubparagraph\subparagraph
\renewcommand{\subparagraph}[1]{\oldsubparagraph{#1}\mbox{}}
\fi

%%% Use protect on footnotes to avoid problems with footnotes in titles
\let\rmarkdownfootnote\footnote%
\def\footnote{\protect\rmarkdownfootnote}

%%% Change title format to be more compact
\usepackage{titling}

% Create subtitle command for use in maketitle
\newcommand{\subtitle}[1]{
  \posttitle{
    \begin{center}\large#1\end{center}
    }
}

\setlength{\droptitle}{-2em}

  \title{}
    \pretitle{\vspace{\droptitle}}
  \posttitle{}
    \author{}
    \preauthor{}\postauthor{}
    \date{}
    \predate{}\postdate{}
  

\usepackage{amsthm}
\newtheorem{theorem}{Theorem}[section]
\newtheorem{lemma}{Lemma}[section]
\theoremstyle{definition}
\newtheorem{definition}{Definition}[section]
\newtheorem{corollary}{Corollary}[section]
\newtheorem{proposition}{Proposition}[section]
\theoremstyle{definition}
\newtheorem{example}{Example}[section]
\theoremstyle{definition}
\newtheorem{exercise}{Exercise}[section]
\theoremstyle{remark}
\newtheorem*{remark}{Remark}
\newtheorem*{solution}{Solution}
\begin{document}

\hypertarget{appendix-appendix}{%
\appendix}


\hypertarget{appendixA}{%
\section{Appendix: Service Order Classification}\label{appendixA}}

Service orders issued by CGU investigated different uses of public
resources in addition to procurement, e.g.~for officials compensation,
for school activities, or for community monitoring of public policies.
The discretion measure advanced here, however, is exclusive to
procurement expenditures made under Law 8,666/93, thus we need to
isolate service orders which investigated procurement processes from the
rest. Since there is no such direct reporting in CGU reports, we
implement a service order classification system based on the information
retrieval and natural-language processing literatures.

We use each service order's description to identify if it is
procurement-related. In these descriptions, CGU auditors report the
purpose of their investigation, e.g.~whether they are looking into
painkiller purchases, whether the municipality has used the funds within
designated goals, or whether primary school teachers were hired for the
implementation of a school program. Using these textual descriptions as
bag-of-words models, we implement a method similar to that of Hopkins \&
King (2009): we stem and combine unigrams to form search patterns that
identify a service order as procurement-related. There are two broad
types of procurement in Law 8,666/93; (i) ordinary procurement of goods
and services, which we call \emph{purchases}; (ii) and procurement of
goods and services used for public works, which we call \emph{works}.
There are different search patterns for each type.

An example is useful for understanding our classification process.
Unigram ``aquisição'' (\emph{acquisition} in English) is stemmed to
``aquisi'' to form a search pattern for purchase-type procurement;
unigrams ``adequação'' and ``habitacional'' are stemmed and combined to
form ``adequa(.)*habitac'' search pattern for works-type procurement --
it picks up variations in main keywords as well as coding mistakes due
to, for instance, multiple whitespace between the two unigrams or when
coding Portuguese special characters (``adequação'' vs ``adequacao'').

\emph{- insert table 1: keywords here -}

The final list contains 19 \emph{n}-grams for identification of
purchases and 17 \emph{n}-grams for works.\footnote{One of these
  keywords in the works search pattern is an ``exclusion keyword,''
  which removes service orders that contain the ``exclusion keyword'' in
  their description from the sample found by the other 17
  \emph{n}-grams.} When any of these words is found, we include the
service order into the purchases or the works group. Since all public
works projects procure goods and services but not all public purchases
are works-related, whenever the search patterns matches service orders
to both groups, we include the service order only in the works group but
not in the purchases group. Public works procurements are a subset of
all public procurements in Brazilian municipalities.

\emph{- insert figure 1: Venn Diagram here -}

As Grimmer \& Stewart (2013) rightly point out, no text analysis
algorithm is perfect and only relying on keyword matches could
potentially lead to misclassification of service orders. Let us suppose
that one description reads ``expenditures made in accordance with
primary education program.'' Using unigram ``expenditure'' would yield a
match for this service order but in fact auditors might be looking at
bonus payments for high-performing teachers. These resources could also
be directed for remodeling schools, in which case we would not know that
this service order should be marked to the works group.

We address these classification issues in three ways: (i) using means
comparison tests of match quality discussed in Assumpcao (2018); (ii)
comparing the same search patterns on another textual description for a
subset of service orders; (iii) finally, comparing the results from the
textual classification against that of procurement-related violations
reported by CGU auditors. The three tests are discussed below.

\hypertarget{quality1}{%
\subsection{Match Quality Measures}\label{quality1}}

The first test on match quality is the means comparison test presented
in Assumpcao (2018), whose reasoning is simple. By adding more
procurement-related terms to the search pattern, we run the risk of
misclassifying service orders as procurement when in fact they are not
-- i.e.~the typical type I error. Ideally, we want to use as few
\emph{n}-grams as possible while still identifying all possible
procurement matches. What Assumpcao (2018) thus suggests is pairwise
means comparison tests between the samples identified by \emph{n} v.
\emph{n-1} keywords. This method translates into a check on whether the
sample identified by one additional keyword is significantly better than
the previous sample with one fewer item.





\emph{- insert table 2: Purchases Search Patterns here -}



\emph{- insert table 2: Works Search Patterns here -}

According to tables \ref{tab2}

\hypertarget{quality2}{%
\subsection{Textual Descriptions}\label{quality2}}

A

\hypertarget{quality3}{%
\subsection{Procurement-related Violations}\label{quality3}}

A

\hypertarget{refs}{}
\leavevmode\hypertarget{ref-AssumpcaotextfindDataDrivenText2018}{}%
Assumpcao, A. (2018). \emph{Textfind: A Data-Driven Text Analysis Tool
for Stata}.

\leavevmode\hypertarget{ref-GrimmerTextDataPromise2013a}{}%
Grimmer, J., \& Stewart, B. M. (2013). Text as Data: The Promise and
Pitfalls of Automatic Content Analysis Methods for Political Texts.
\emph{Political Analysis}, \emph{21}(3), 267--297.
\url{https://doi.org/10.1093/pan/mps028}

\leavevmode\hypertarget{ref-HopkinsMethodAutomatedNonparametric2009}{}%
Hopkins, D., \& King, G. (2009). A Method of Automated Nonparametric
Content Analysis for Social Science. \emph{American Journal of Political
Science}, \emph{54}(1), 229--247.
\url{https://doi.org/10.1111/j.1540-5907.2009.00428.x}


\end{document}
